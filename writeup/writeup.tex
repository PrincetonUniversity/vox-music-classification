\documentclass{article}

\input{defs}

\title{Music Genre Classification\\\large COS424}
\author{Vladimir Feinberg, Siddhartha Jayanti}
\date{9 February 2016}

\begin{document}
\maketitle

\section{Abstract}

 blah blah \cite{davis2012fractional}.
 
 \section{Introduction}

A basic problem in the field of music is to label a song into a {\em genre} (jazz, rock, country, classical etc.).
In general, a given song can have multiple genres, and two people may disagree on whether a song $s$ can be labeled with a given genre $g$.
Furthermore, there are a huge number of genres and some are very similar to others.
Thus, the most general version of the {\em genre classification problem} which intuitively asks for a complete list of genres given a song is hard to formalize
and solve.
In this paper, we will be discussing a simpler version of this problem which we call {\em Genre-Classify} that asks for a single genre label given a song.
Our given data set contains 1000 different songs, which we shall call our song set $S$.
Each song $s \in S$ is exactly 30 seconds long and is labeled with a single genre label 
$g(s) \in G = \{ \textit{blues, classical, country, disco, hip-hop, jazz, metal, pop, reggae, rock} \}$.
There are exactly 100 songs of each genre.
Our problem can be formally stated as follows:
Given a random subset of pairs $P = \{(s, g(s)) | s \in T \}$ where $T \subsetneq S$, derive a function $g': T' = (S - T) \rightarrow G$ 
that attempts to maximize the proportion of correct guesses $g'(s) = g(s), s \in T'$.
In our experiments we let our {\em training set} $T$ be picked as a random subset of size $0.8 |S|$,
and test the models $g'$ we derive on our {\em test set} $T'$.

One problem in using algorithms on the music data, is the dimensionality of the data (if the song was interpreted as a vector of bits for instance).
One way to reduce the dimensionality of the problem would be to take the first few components of a PCA; 
but better methods are possible when we have a labelled data set \cite{WellingNote}.
One such method is to encode musical {\em features} in a vector using the Fisher Linear Discriminant Analysis (Fisher-LDA) method \cite{WellingNote}.
This method is used on {\em timbre-related} musical features in Enrique et al. \cite{ERLGRR}.
They obtain a $4.09\%$ probability of error when they use the MFCC and Fisher-LDA.
A later paper by Chang et al. \cite{CJI10} uses both short and long term features of music and the {\em compressive sampling} technique to keep
the dimensionality low.
They obtain $92.7\%$ accuracy when they use multiple short and long term features.

\bibliographystyle{acm}
\bibliography{bibliography}

\end{document}